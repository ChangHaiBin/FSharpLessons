\documentclass[12pt]{article}

\usepackage{amsmath, amsfonts, amsthm}
\usepackage{enumerate}
\usepackage{hyperref}
\usepackage{graphicx}
\usepackage{geometry}
\geometry{
	a4paper,
 	left=24mm,
 	right=24mm,
 	top=30mm,
 	bottom=35mm
}
\usepackage{color}
\definecolor{bluekeywords}{rgb}{0.13,0.13,1}
\definecolor{greencomments}{rgb}{0,0.5,0}
\definecolor{redstrings}{rgb}{0.9,0,0}
\definecolor{cyantypes}{RGB}{0,183,235}

\usepackage{listings}
\lstdefinelanguage{FSharp}%
{morekeywords=[1]{let, new, match, with, rec, open, module, namespace, type, of, member, % 
and, for, while, true, false, in, do, begin, end, fun, function, return, yield, try, %
mutable, if, then, else, cloud, async, static, use, abstract, interface, inherit, finally },
  morekeywords=[2]{double, float, int, string, List, BigInteger},
  otherkeywords={ let!, return!, do!, yield!, use!, var, select, where, order, by },
% otherkeywords={from}
  keywordstyle=[1]\color{bluekeywords},
  keywordstyle=[2]\color{cyantypes},
  sensitive=true,
  basicstyle=\ttfamily,
	breaklines=true,
  xleftmargin=\parindent,
  aboveskip=\bigskipamount,
	tabsize=4,
  morecomment=[l][\color{greencomments}]{///},
  morecomment=[l][\color{greencomments}]{//},
  morecomment=[s][\color{greencomments}]{{(*}{*)}},
  morestring=[b]",
  showstringspaces=false,
  literate={`}{\`}1,
  stringstyle=\color{redstrings},
}
\definecolor{codegreen}{rgb}{0,0.6,0}
\definecolor{codegray}{rgb}{0.5,0.5,0.5}
\definecolor{codepurple}{rgb}{0.58,0,0.82}
\definecolor{backcolour}{rgb}{0.95,0.95,0.92}
\lstdefinestyle{mystyle}{
    backgroundcolor=\color{backcolour},   
    commentstyle=\color{codegreen},
    keywordstyle=\color{magenta},
    numberstyle=\tiny\color{codegray},
    stringstyle=\color{codepurple},
    basicstyle=\footnotesize\ttfamily,
    breakatwhitespace=false,         
    breaklines=true,                 
    captionpos=b,                    
    keepspaces=true,                 
    numbers=left,                    
    numbersep=5pt,                  
    showspaces=false,                
    showstringspaces=false,
    showtabs=false,                  
    tabsize=4
}
\lstset{style=mystyle}

\newtheorem*{question*}{Question}
\newtheorem*{modQuestion*}{Modified Question}
\newtheorem*{origQuestion*}{Original Question}
\begin{document}


\begin{center}

{\large F\# Tutorial\\} \vspace{2mm}
\textbf{\LARGE Pipe-Forward Operator}\\
\vspace{1.5mm}
{\Large\emph{\today}}

\end{center}


\section{Modified Project Euler Questions}
\begin{flushleft}

\url{https://projecteuler.net/problem=1}

\url{https://projecteuler.net/problem=2}
\end{flushleft}

Change the url so that you get \texttt{problem=3, problem=4}, etc.

\subsection*{Question 1} 


\begin{origQuestion*}
Implement a function that sums up all multiples of $3$ or $5$ in a list.
\end{origQuestion*}

\subsection*{Question 2} 
\begin{origQuestion*}
The Fibonacci sequence (starting with $1$ and $2$) looks something like:
\[
1, 2, 3, 5, 8, 13, 21, 34, 55, 89, \ldots
\]
(For example, $1 + 2 = 3, 2 + 3 = 5, 3 + 5 = 8,$ etc.)

Find the sum of all even-valued fibonacci numbers below $4$ million.
\end{origQuestion*}
\begin{enumerate}[(a)]
\item List down the first $40$ fibonacci numbers. Show that the $40$th Fibonacci number already exceed $4$ million (in fact, probably the $32$nd or $33$rd number already exceed $4$ million)
\item Sum all even-valued fibonacci numbers below $4$ million. (In particular, by part (a), the first $40$ numbers are already sufficient)
\end{enumerate}
\subsection*{Question 3} 
\subsubsection*{Exercise (Euler Project Question 3)}

\begin{modQuestion*}
Write a function that takes a list of (positive) integers, and returns the largest prime number in that list (assuming each integer is less than \texttt{INT\_MAX}).
\end{modQuestion*}

\subsection*{Question 4} 
\begin{origQuestion*}
A palindromic number reads the same from left-to-right or right-to-left. 

The largest palindromic number made from the product of two 2-digit numbers is $9009 = 91 \times 99$.

Find the largest palindrome made from the product of two 3-digit numbers.
\end{origQuestion*}

\subsection*{Question 5} 

\begin{modQuestion*}
Given a list of integers, find the lowest common multiple (LCM) of all those numbers. (Assume no integer overflow)
\end{modQuestion*}
\subsection*{Question 6} 
\begin{origQuestion*}
Given a list of integers $x_1, x_2, \ldots, x_n$, write a function that calculates the following:
\[
\left(\sum_{i=1}^n x_i\right)^2 - \left(\sum_{i=1}^n {x_i}^2\right)
\]
\end{origQuestion*}
\subsection*{Question 7} 

\begin{origQuestion*}
The list of prime numbers are $2, 3, 5, 7, 11, 13, \ldots$. We can see that the $6$th prime number is $13$. 

What is the $10001$th prime number?
\end{origQuestion*}

\begin{enumerate}[(a)]
\item How many prime numbers are there between $2$ and $500000$? Verify that there are more than $10000$ prime numbers between this range. (In fact, more than $40000$ prime numbers)
\item What is the $10001$th prime number between $2$ and $500000$?
\end{enumerate}
\subsection*{Question 8} 

\begin{modQuestion*}
Given a list of digits, find four adjacent digits with the largest product. For example, in the following number:
\[
7316717653133062491922511\mathbf{9674}426574742355349194934
\]
The 4 consecutive digits that gives the largest product is $9 \times 6 \times 7 \times 4 = 9674$

(Notice that this line is the first line in the original question)
\end{modQuestion*}
\subsection*{Question 9} 

\begin{origQuestion*}
Find the only Pythagorean triplet $a, b, c$ that satisfy:
\[
a < b < c, \hspace{1.0cm} a + b + c = 1000, \hspace{1.0cm} a^2 + b^2 = c^2
\]
\end{origQuestion*}
\subsection*{Question 10} 

\subsubsection*{Exercise (Euler Project Question 10)}

\begin{modQuestion*}
Given a number $N < 200,000$, find the sum of all prime numbers between $2$ and $N$.
\end{modQuestion*}


\pagebreak
\section{Original Project Euler Solutions}
\subsection*{Question 1, 2, 4, 6, 7, 9} 
We did not modify Question 1, 2, 4, 6, 7, 9.
\subsection*{Question 5} 
\begin{origQuestion*}
Find the least common multiple (LCM) of $1$ to $20$. (You may encounter integer overflow)
\end{origQuestion*}
\subsection*{Question 8} 
\begin{origQuestion*}
In the webpage, a $1000-digit$ number is provided.

The four adjacent digits in the $1000$-digit number that have the greatest product are $9 \times 9 \times 8 \times 9 = 5832$.
\begin{center}
\texttt{731671765313......31998900.......2963450}
\end{center}
Find the thirteen adjacent digits in the $1000$-digit number that have the greatest product. (You may encounter integer overflow)
\end{origQuestion*}
\subsection*{Question 10} 
\begin{origQuestion*}
The sum of the primes below $10$ is $2 + 3 + 5 + 7 = 17$

Find the sum of all the primes below two million $(2,000,000)$. (You may encounter integer overflow)
\end{origQuestion*}


\vfill
\pagebreak


\subsection*{Question 3} 
\begin{question*}
Given an integer $Z$, write a function that finds the largest prime factor of $Z$. e.g. The prime factors of $13195$ are $5, 7, 13, 29$, and so the largest for $13195$ is $29$.
\end{question*}

\subsubsection*{Problem Analysis}
Remark: Given an integer $Z$, it is possible that the largest prime factor of $Z$ is greater than $\sqrt{Z}$
\begin{itemize}

\item Example: $6 \times 11 = 66$. The largest prime factor is $11 > \sqrt{66} \approx 8.12$.
\end{itemize}
To solve this question, we need some additional mathematical consideration (which is not quite directly related to programming).
\begin{enumerate}[(a)]
\item Let $S_1 = \{a_1, \ldots, a_n\}$ be all the factors of $Z$ (not necessarily prime factors) between $1$ and $\sqrt{Z}$. This set will always contain at least one element: $a_1 = 1$.
\item Let $S_2 = \left\{\dfrac{Z}{a_1},\ldots, \dfrac{Z}{a_n}\right\}$. These are all the factors of $Z$ between $\sqrt{Z}$ and $Z$. This set will always contain at least one element: $\dfrac{Z}{a_1} = Z$.
\item So, $S_1 \cup S_2 =  \left\{a_1, \ldots, a_n, \dfrac{Z}{a_1},\ldots, \dfrac{Z}{a_n}\right\}$ are all the factor of $Z$ (not necessarily prime factors).
\item Out of our list of candidates $S_1 \cup S_2$, which number is the \underline{largest}, \underline{prime} number?
\end{enumerate}

\end{document}